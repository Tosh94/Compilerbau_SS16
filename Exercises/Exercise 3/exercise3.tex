Let $G = \langle N, \Sigma , P, S \rangle \in CFG_\Sigma$ where $P = \{\pi_1, \dots ,  \pi_p \}$, $X = N \cup \Sigma$ with $G \in LL(1)$.\\
Claim: Then $G$ is \emph{unambiguous}.\\[1ex]
Proof by contradiction:\\
Let $G$ be \emph{ambiguous}. Then for some $w \in \Sigma^{*}$ there are 2 different analyses $z$ and $z'$,\\
$S \derive{z}{l} w, S \derive{z'}{l} w, z \neq z', z, z' \in [p]^{*}$ where \\
$z = kil, z' = kjm, l \neq m, i \neq j, k, l, m \in [p]^{*}, i, j \in [p]$.\\
Then $S \derive{k}{l} vA\alpha , A \in N, v \in \Sigma^{*}, \alpha \in X^{*}$ and\\
\begin{equation*}\begin{split}
vA\alpha \derive{i}{l} v\beta\alpha \derive{l}{l} vx = w, x \in \Sigma^{*}\\
vA\alpha \derive{j}{l} v\gamma\alpha \derive{m}{l} vy = w, y \in \Sigma^{*}\\
\implies x = y
\end{split}\end{equation*}
Since $x = Y$, $first_1(x) = first_1(y)$ and thus $G \not\in LL(1)$. \quad\Lightning\\[1ex]
By the principle of contradiction follows that $G$ is \emph{unambiguous}.