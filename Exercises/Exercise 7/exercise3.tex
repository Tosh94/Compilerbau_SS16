\subsection*{a)}

$D_{S' \rightarrow S}$:

\begin{center}
\begin{tikzpicture}[align=center,on grid,auto,node distance=1.5cm and 1cm]
   %Syntax:
   %\node[align=<alignment>] (<name>) [<location>] {<text>};
   \node[align=center] (src) [] {S'};
   \node[align=center,rounded corners=0.15cm,draw=black] (si2) [left=of src]  {i2};
   \node[align=center,rounded corners=0.15cm,draw=black] (si1) [left=of si2]  {i1};
   \node[align=center,rounded corners=0.15cm,draw=black] (ss2) [right=of src]  {s2};
   \node[align=center] (dst) [below=of src] {S};
   \node[align=center,rounded corners=0.15cm,draw=black] (di1) [left=of dst]  {i1};
   \node[align=center,rounded corners=0.15cm,draw=black] (di2) [left=of di1]  {i2};
   \node[align=center,rounded corners=0.15cm,draw=black] (ds1) [right=of dst]  {s1};
   \node[align=center,rounded corners=0.15cm,draw=black] (ds2) [right=of ds1]  {s2};
   
   %Syntax:
   %\path[<line type>]
   % (<name of start>) edge [<textlocation>, <curve type>] {<text>} (<name of target>)
    \path[-]
    (src) edge [dashed] (dst)
      ;
    \path[->]
    (si1) edge [] (di2)
    (ds1) edge [bend right=30] (di1)
    (ds2) edge [] (ss2)
      ;
\end{tikzpicture}
\end{center}
$D_{S \rightarrow AA}$:

\begin{center}
\begin{tikzpicture}[align=center,on grid,auto,node distance=1.5cm and 1cm]
   %Syntax:
   %\node[align=<alignment>] (<name>) [<location>] {<text>};
   \node[align=center] (src) [] {S'};
   \node[align=center,rounded corners=0.15cm,draw=black] (si2) [left=of src]  {i2};
   \node[align=center,rounded corners=0.15cm,draw=black] (si1) [left=of si2]  {i1};
   \node[align=center,rounded corners=0.15cm,draw=black] (ss1) [right=of src]  {s1};
   \node[align=center,rounded corners=0.15cm,draw=black] (ss2) [right=of ss1]  {s2};
   \node[align=center] (dst1) [below left=1.5cm and 3.5 cm of src] {A};
   \node[align=center,rounded corners=0.15cm,draw=black] (d1i1) [left=of dst1]  {i1};
   \node[align=center,rounded corners=0.15cm,draw=black] (d1i2) [left=of d1i1]  {i2};
   \node[align=center,rounded corners=0.15cm,draw=black] (d1s1) [right=of dst1]  {s1};
   \node[align=center,rounded corners=0.15cm,draw=black] (d1s2) [right=of d1s1]  {s2};
   \node[align=center] (dst2) [below right=1.5cm and 3.5 cm of src] {A};
   \node[align=center,rounded corners=0.15cm,draw=black] (d2i1) [left=of dst2]  {i1};
   \node[align=center,rounded corners=0.15cm,draw=black] (d2i2) [left=of d2i1]  {i2};
   \node[align=center,rounded corners=0.15cm,draw=black] (d2s2) [right=of dst2]  {s2};
   
   %Syntax:
   %\path[<line type>]
   % (<name of start>) edge [<textlocation>, <curve type>] {<text>} (<name of target>)
    \path[-]
    (src) edge [dashed] (dst1)
    (src) edge [dashed] (dst2)
      ;
    \path[->]
    (si1) edge [] (d1i2)
    (si2) edge [] (d2i2)
    (d1s1) edge [bend right=30] (d1i1)
    (d1s2) edge [] (ss1)
    (d2s2) edge [] (ss2)
      ;
\end{tikzpicture}
\end{center}
\bigskip

$D_{A \rightarrow a}$:

\begin{center}
\begin{tikzpicture}[align=center,on grid,auto,node distance=1.5cm and 1cm]
   %Syntax:
   %\node[align=<alignment>] (<name>) [<location>] {<text>};
   \node[align=center] (src) [] {S};
   \node[align=center,rounded corners=0.15cm,draw=black] (si2) [left=of src]  {i2};
   \node[align=center,rounded corners=0.15cm,draw=black] (ss2) [right=of src]  {s2};
   \node[align=center] (dst) [below=of src] {A};
   \node[align=center,rounded corners=0.15cm,draw=black] (di1) [left=of dst]  {i1};
   \node[align=center,rounded corners=0.15cm,draw=black] (di2) [left=of di1]  {i2};
   \node[align=center,rounded corners=0.15cm,draw=black] (ds2) [right=of dst]  {s2};
   
   %Syntax:
   %\path[<line type>]
   % (<name of start>) edge [<textlocation>, <curve type>] {<text>} (<name of target>)
    \path[-]
    (src) edge [dashed] (dst)
      ;
    \path[->]
    (si2) edge [] (di2)
    (ds2) edge [] (ss2)
      ;
\end{tikzpicture}
\end{center}
\bigskip
\begin{minipage}{0.5\textwidth} 
$D_{A \rightarrow b}$:


\begin{center}
\begin{tikzpicture}[align=center,on grid,auto,node distance=1.5cm and 1cm]
   %Syntax:
   %\node[align=<alignment>] (<name>) [<location>] {<text>};
   \node[align=center] (src) [] {A};
   \node[align=center,rounded corners=0.15cm,draw=black] (si1) [left=of src]  {i1};
   \node[align=center,rounded corners=0.15cm,draw=black] (ss1) [right=of src]  {s1};
   \node[align=center,rounded corners=0.15cm,draw=black] (ss2) [right=of ss1]  {s2};
   \node[align=center] (dst) [below=of src] {a};
   
   %Syntax:
   %\path[<line type>]
   % (<name of start>) edge [<textlocation>, <curve type>] {<text>} (<name of target>)
    \path[-]
    (src) edge [dashed] (dst)
      ;
    \path[->]
    (si1) edge [bend right=30] (ss2)
      ;
\end{tikzpicture}
\end{center}
\end{minipage}
\begin{minipage}{0.5\textwidth} 
$D_{A \rightarrow b}$:


\begin{center}
\begin{tikzpicture}[align=center,on grid,auto,node distance=1.5cm and 1cm]
   %Syntax:
   %\node[align=<alignment>] (<name>) [<location>] {<text>};
   \node[align=center] (src) [] {A};
   \node[align=center,rounded corners=0.15cm,draw=black] (si2) [left=of src]  {i2};
   \node[align=center,rounded corners=0.15cm,draw=black] (ss1) [right=of src]  {s1};
   \node[align=center,rounded corners=0.15cm,draw=black] (ss2) [right=of ss1]  {s2};
   \node[align=center] (dst) [below=of src] {b};
   
   %Syntax:
   %\path[<line type>]
   % (<name of start>) edge [<textlocation>, <curve type>] {<text>} (<name of target>)
    \path[-]
    (src) edge [dashed] (dst)
      ;
    \path[->]
    (si2) edge [bend right=30] (ss1)
      ;
\end{tikzpicture}
\end{center}
\end{minipage}

\subsection*{b)}


\begin{equation*}
\begin{split}
IS(A) &= \{\{(i1, s2)\},\{(i2, s1)\}\} \\
IS(S) &= \{\emptyset\} \\
IS(S') &= \{\emptyset\}
\end{split}
\end{equation*}

No, $G$ is not circular, since the Production of $S$ does not provide any dependency from inherited to synthetic attributes. A cycle could only be achieved by utilizing both Productions of $A$ after applying $S \rightarrow AA$ on the left branch, which is no valid syntax tree.